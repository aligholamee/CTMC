\chapter{Problem Definition}

The emerging need of image processing techniques in everyday life is inevitable. There have been lots of image processing algorithms proposed to increase the overall availability of tools in image understanding and using these tools to improve the human life. \textit{Template Matching} is a task that its applications are obvious in everyday life. Computer vision tasks such as \textit{object detection}, \textit{object recognition} and other tasks based on these two main questions; Is there a certain object in the image? And if yes, where is it in the image?, can be classified as subproblems of \textit{template matching} task. This task can be extended to counting the number of occurrences of an object in a given image. Given a main image and a template image, we want to find occurrences of the template image in main image. This problem can be solved in many ways. In the next chapter, we'll analyze the procedure of \textit{naive} template matching to solve this problem. In further chapters, we'll provide the procedure of using \textit{Fast Fourier Transform} to convert the images into frequency domain. We'll show that finding the maximum value of the result of the convolution of two images is where the template has occurred in main image.